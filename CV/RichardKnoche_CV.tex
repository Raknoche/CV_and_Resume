\documentclass[a4paper,10pt]{article}

%A Few Useful Packages
\usepackage{marvosym}
\usepackage{fontspec} 					%for loading fonts
\usepackage{xunicode,xltxtra,url,parskip} 	%other packages for formatting
\RequirePackage{color,graphicx}
\usepackage[usenames,dvipsnames]{xcolor}
\usepackage[big]{layaureo} 				%better formatting of the A4 page
% an alternative to Layaureo can be ** \usepackage{fullpage} **
\usepackage{supertabular} 				%for Grades
\usepackage{titlesec}					%custom \section

%Setup hyperref package, and colours for links
\usepackage{hyperref}
\definecolor{linkcolour}{rgb}{0,0.2,0.6}
\hypersetup{colorlinks,breaklinks,urlcolor=linkcolour, linkcolor=linkcolour}

%FONTS
\defaultfontfeatures{Mapping=tex-text}
%\setmainfont[SmallCapsFont = Fontin SmallCaps]{Fontin}
%%% modified for Karol Kozioł for ShareLaTeX use
\setmainfont[
SmallCapsFont = Fontin-SmallCaps.otf,
BoldFont = Fontin-Bold.otf,
ItalicFont = Fontin-Italic.otf
]
{Fontin.otf}
%%%

%CV Sections inspired by: 
%http://stefano.italians.nl/archives/26
\titleformat{\section}{\Large\scshape\raggedright}{}{0em}{}[\titlerule]
\titlespacing{\section}{0pt}{3pt}{3pt}
%Tweak a bit the top margin
%\addtolength{\voffset}{-1.3cm}

%Italian hyphenation for the word: ''corporations''
\hyphenation{im-pre-se}

%-------------WATERMARK TEST [**not part of a CV**]---------------
\usepackage[absolute]{textpos}

\setlength{\TPHorizModule}{30mm}
\setlength{\TPVertModule}{\TPHorizModule}
\textblockorigin{2mm}{0.65\paperheight}
\setlength{\parindent}{0pt}

%--------------------BEGIN DOCUMENT----------------------
\begin{document}

%WATERMARK TEST [**not part of a CV**]---------------
%\font\wm=''Baskerville:color=787878'' at 8pt
%\font\wmweb=''Baskerville:color=FF1493'' at 8pt
%{\wm 
%	\begin{textblock}{1}(0,0)
%		\rotatebox{-90}{\parbox{500mm}{
%			Typeset by Alessandro Plasmati with \XeTeX\  \today\ for 
%			{\wmweb \href{http://www.aleplasmati.comuv.com}{aleplasmati.comuv.com}}
%		}
%	}
%	\end{textblock}
%}

\pagestyle{empty} % non-numbered pages

\font\fb=''[cmr10]'' %for use with \LaTeX command

%--------------------TITLE-------------
\par{\centering
		{\Huge \textsc{Richard Knoche}
	}\bigskip\par}

%--------------------SECTIONS-----------------------------------
%Section: Personal Data
\section{Personal Information}

\begin{tabular}{rl}
    \textsc{Address:}   & 8810 62nd Ave, Berwyn Heights, Maryland 20740 \\
    \textsc{Phone:}     & (703)-801-4456\\
    \textsc{email:}     & \href{mailto:raknoche@gmail.com}{raknoche@gmail.com}
\end{tabular}

%Section: Education
\section{Education}
\begin{tabular}{rl}	
 \textsc{Expected} 2016 & Doctor of Philosophy (Ph.D.) in \textsc{Physics}, \textbf{University of Maryland}\\
& Thesis: ``Signal Corrections and Calibrations in the LUX Dark Matter Detector'' \\&\\
\textsc{May} 2011& Bachelor of Science (B.S.) in \textsc{Physics}, magna cum laude,  \textbf{James Madison University} 
\end{tabular}

%Section: Experience at the top
\section{Experience}
\begin{tabular}{r|p{11cm}}
 \multicolumn{2}{c}{} \\
 \textsc{Aug 2011 - Present} & Ph.D. Candidate at \textsc{The University of Maryland} \\&\emph{The LUX Dark Matter Detector}\\&\footnotesize{
 \begin{itemize}
 
 \item Worked with an international collaboration to produce the world's most sensitive WIMP-nucleon scattering cross section limits.  
 
\item Designed bench-top experiments to test the safe deployment and removal of tritiated methane in liquid xenon detectors.  

\item Successfully created and carried out a tritiated methane calibration protocol for the LUX detector.  

\item Used tritium data to produce the world's most precise electron recoil calibration in a liquid xenon TPC.  This technique has been adopted by similar experiments in Asia and Europe. 

\item Developed novel techniques to produce 3D position dependent signal corrections in a spatially and time dependent electric field.  

\item Automated the extraction of hundreds of data features from krypton calibration data.  

\item Calibrated the detector's combined energy model.

\item Helped maintain a GEANT4 based simulations framework.

\item Built on-site gas sampling system for real-time monitoring of xenon impurities at the part-per-trillion level.

\item Directed on-site detector operations as Deputy Science Coordination Manager.
\end{itemize}
}\\ 

 \multicolumn{2}{c}{} \\
 \textsc{June-Sep 2010} & Research Assistant at \textsc{Goddard Space Flight Center} \\&\emph{SWIFT Gamma-Ray Burst Mission}\\&\footnotesize{
 \begin{itemize}
\item Analyzed data from Swift Burst Alert Telescope (BAT) to search for hard X-ray emissions around the on-set time of supernovae.
\item Quantified the X-ray counterpart to Fermi-LAT pulsar observations using X-ray emission data from the Chandra and BeppoSax missions.
\end{itemize}
 }\\
 \end{tabular}
\begin{tabular}{r|p{11cm}}
 
 \multicolumn{2}{c}{} \\
\textsc{Aug 2008 - May 2011} & Research Assistant at \textsc{James Madison University}\\&\emph{Department of Physics and Astronomy}\\&\footnotesize{
\begin{itemize}

\item Designed and maintained table-top experiments to characterize the complex, non-linear behavior of granular systems.
\item Implemented computer vision techniques to quantitatively characterize particle movement in a two dimensional shear flow.
\item Utilized optical polarization techniques to quantify stress networks in granular systems.
\item Performed statistical analysis to characterize relevant parameters in a two dimensional granular shear flow.

\end{itemize}
}\\
\end{tabular}

%Section: Publications
\section{Refereed Publications}
\begin{enumerate}  

\item D.~S.~Akerib {\it et al.} [LUX Collaboration], ``Results from a search for dark matter in LUX with 332 live days of exposure,'' arXiv preprint arXiv:1608.07648 (2016). Submitted to Phys. Rev. Lett.

\item D.~S.~Akerib {\it et al.} [LUX Collaboration],
  ``Low-energy (0.7-74 keV) nuclear recoil calibration of the LUX dark matter experiment using DD neutron scattering kinematics,'' arXiv preprint arXiv:1608.05381 (2016). Submitted to Phys. Rev. C.

\item D.~S.~Akerib {\it et al.} [LUX Collaboration],
  ``Chromatographic separation of radioactive noble gases from xenon,''
  arXiv preprint arXiv:1605.03844 (2016). Submitted to Astropart. Phys.
  
\item   D.~S.~Akerib {\it et al.} [LUX Collaboration],
  ``Results on the Spin-Dependent Scattering of Weakly Interacting Massive Particles on Nucleons from the Run 3 Data of the LUX Experiment,''
  Phys.\ Rev.\ Lett.\  {\bf 116}, no. 16, 161302 (2016)
  
\item   D.~S.~Akerib {\it et al.} [LUX Collaboration],
  ``Improved Limits on Scattering of Weakly Interacting Massive Particles from Reanalysis of 2013 LUX Data,''
  Phys.\ Rev.\ Lett.\  {\bf 116}, no. 16, 161301 (2016)
  %%CITATION = doi:10.1103/PhysRevLett.116.161301;%%
  %134 citations counted in INSPIRE as of 05 Aug 2016
  
\item   D.~S.~Akerib {\it et al.} [LUX Collaboration],
  ``Tritium calibration of the LUX dark matter experiment,''
  Phys.\ Rev.\ D {\bf 93}, no. 7, 072009 (2016)
  %%CITATION = doi:10.1103/PhysRevD.93.072009;%%
  %8 citations counted in INSPIRE as of 05 Aug 2016
  
\item   D.~S.~Akerib {\it et al.} [LUX Collaboration],
  ``FPGA-based Trigger System for the LUX Dark Matter Experiment,''
  Nucl.\ Instrum.\ Meth.\ A {\bf 818}, 57 (2016)
  %%CITATION = doi:10.1016/j.nima.2016.02.017;%%
  
\item   D.~S.~Akerib {\it et al.},
  ``Radiogenic and Muon-Induced Backgrounds in the LUX Dark Matter Detector,''
  Astropart.\ Phys.\  {\bf 62}, 33 (2015)
  %%CITATION = doi:10.1016/j.astropartphys.2014.07.009;%%
  %27 citations counted in INSPIRE as of 05 Aug 2016
  
\item   D.~S.~Akerib {\it et al.} [LUX Collaboration],
  ``First results from the LUX dark matter experiment at the Sanford Underground Research Facility,''
  Phys.\ Rev.\ Lett.\  {\bf 112}, 091303 (2014)
  %%CITATION = doi:10.1103/PhysRevLett.112.091303;%%
  %1328 citations counted in INSPIRE as of 05 Aug 2016

\item   D.~S.~Akerib {\it et al.} [LUX Collaboration],
  ``The Large Underground Xenon (LUX) Experiment,''
  Nucl.\ Instrum.\ Meth.\ A {\bf 704}, 111 (2013)
  %%CITATION = doi:10.1016/j.nima.2012.11.135;%%
  %159 citations counted in INSPIRE as of 05 Aug 2016
 
\item   D.~S.~Akerib {\it et al.} [LUX Collaboration],
  ``Technical Results from the Surface Run of the LUX Dark Matter Experiment,''
  Astropart.\ Phys.\  {\bf 45}, 34 (2013)
  %%CITATION = doi:10.1016/j.astropartphys.2013.02.001;%%
  %42 citations counted in INSPIRE as of 05 Aug 2016
 
\end{enumerate}

\section{Conference Proceedings}
\begin{enumerate}

\item A.~Murphy {\it et al.} [LUX Collaboration], ``The LUX direct dark matter search,'' AIP Conf. Proc. {\bf 1743}, 050012 (2016)

\item  C.~Carmona-Benitez {\it et al.} [LUX Collaboration], ``First Results of the LUX Dark Matter Experiment,''
  Nucl.\ Part.\ Phys.\ Proc.\  {\bf 273-275}, 309 (2016).
  %%CITATION = doi:10.1016/j.nuclphysbps.2015.09.043;%%

\item  M.~Moongweluwan {\it et al.} [LUX Collaboration], ``The impact of photon flight path on S1 pulse shape analysis in liquid xenon two-phase detectors'' JINST {\bf 11}, no. 02, C02036 (2016)
    
\item  D.~Akerib {\it et al.} [LUX Collaboration], ``The LUX Experiment,'' Phys.\ Procedia {\bf 61}, 74 (2015).
  
\item M.~Horn {\it et al.} [LUX Collaboration], ``Results from the LUX dark matter experiment,'' Nucl.\ Instrum.\ Meth.\ A {\bf 784}, 504 (2015).

  \item   A.~Bradley {\it et al.}  [LUX Collaboration], ``Radon-related Backgrounds in the LUX Dark Matter Search,'' Phys.\ Procedia {\bf 61}, 658 (2015).
   
  \item C.~Faham {\it et al.}  [LUX Collaboration], ``First Dark Matter Search Results from the Large Underground Xenon (LUX) Experiment,''
  arXiv:1405.5906 (2014).
  
  \item   M.~Szydagis {\it et al.} [LUX Collaboration], ``A Detailed Look at the First Results from the Large Underground Xenon (LUX) Dark Matter Experiment,''  arXiv:1402.3731 (2014).

   \item V.~Gehman {\it et al.} [LUX Collaboration], ``Direct Search for Dark Matter with Two-Phase XENON Detectors: Current Status of Lux and Plans for LZ,'' Frascati Phys.\ Ser.\  {\bf 58}, 51 (2014).


 \item M.~Woods {\it et al.} [LUX Collaboration], ``Underground Commissioning of LUX,'' arXiv:1306.0065 (2013)

 \item S.~Fiorucci {\it et al.} [LUX Collaboration], ``The LUX Dark Matter Search – Status Update,'' J.\ Phys.\ Conf.\ Ser.\  {\bf 460}, 012005 (2013)

 \end{enumerate}
  
  
  
%Section: Conference Presentations
\section{Conference Presentations}
\begin{itemize}

\item ``Development of the LUX detector's CH3T calibration source and ER response,'' APS April Meeting, Baltimore, Maryland (April 2015)

\item ``Search for the X-ray Counterpart of Pulsars with GeV Emissions,'' American \newline Astronomical Society Meeting, Seattle, Washington (Jan 2011)

\end{itemize}

%Section: Professional Organizations
\section{Professional Organizations}
\begin{itemize}

\item American Physical Society

\end{itemize}


%Section: Honors
\section{Honors}
\begin{tabular}{rl}
 \textsc{August} 2010 & John Mather Nobel Scholar Award \\
\textsc{March} 2010 & Henry W. Leap Scholarship \\
\textsc{March} 2010 & Sigma Pi Sigma \\
2009-2011 & President's List, James Madison University \\
2008-2011 & Dean's List, James Madison University \\
\end{tabular}


%\hypertarget{gmat}{\textsc{Gmat}\setmainfont{LMRoman10 Regular}\textregistered\setmainfont[SmallCapsFont=Fontin-SmallCaps]{Fontin-Regular}}

%\XeTeXpdffile ''GMAT.pdf'' page 1 scaled 800

\end{document}
